First, a bit of terminology\+:
\begin{DoxyItemize}
\item feature\+: Some bit of data considered to be useful for analysis of game play. Usually calculated from game event data.
\item per-\/level feature\+: A feature that records data about what a player did over the course of a level in the game.
\item per-\/custom-\/count feature\+: A feature that records data about some {\itshape thing} that may have multiple instances over a gameplay session. A common example would be question-\/answer prompts.
\item aggregate feature\+: A feature which records data across an entire gameplay session (as opposed to an individual level, for example).
\item raw csv\+: A csv file containing raw game event data. More-\/or-\/less a database dump, but with J\+S\+ON objects split across columns.
\item processed csv\+: A csv file primarily containing feature data. This typically includes a great many per-\/level features (i.\+e. one instance of each feature per game level), a few per-\/custom-\/count features, and a moderate number of aggregate features.
\end{DoxyItemize}

In order to add a new game to the feature extraction tool, complete the following steps\+:


\begin{DoxyEnumerate}
\item First, we must define some things about the data we are extracting. We do this in a J\+S\+ON file, under the {\bfseries{schemas/\+J\+S\+ON}} folder. By convention, the name of the J\+S\+ON file should be the same as the game ID used in the database.
\end{DoxyEnumerate}

e.\+g. For the \char`\"{}\+Wave\char`\"{} game, the database uses an app\+\_\+id of {\ttfamily W\+A\+V\+ES}, so we name the J\+S\+ON schema file as {\bfseries{W\+A\+V\+E\+S.\+json}} A J\+S\+ON schema file has three elements\+:
\begin{DoxyItemize}
\item {\ttfamily db\+\_\+columns}\+: This element should be a dictionary mapping the names of each column in the database table to a string describing the column.
\item {\ttfamily events}\+: This element should be a dictionary mapping names of event types to sub-\/dictionaries defining the data in the events.
\begin{DoxyItemize}
\item These sub-\/dictionaries are similar to the db\+\_\+columns dictionary. They map each property name for a given event type to the type of that property.
\end{DoxyItemize}
\item {\ttfamily features}\+: This element should in turn contain three elements\+:
\begin{DoxyItemize}
\item {\ttfamily perlevel}\+: This sub-\/element should be a dictionary mapping the names of per-\/level features to descriptions of how the features are calculated.
\item {\ttfamily per\+\_\+custom\+\_\+count}\+: This sub-\/element should be a dictionary mapping the names of features which are repeated for some specific number of times to a subdictionary. This, again, has three elements\+:
\begin{DoxyItemize}
\item {\ttfamily count}\+: The number of times the feature is repeated
\item {\ttfamily prefix}\+: The prefix to use to distinguish repeats of the feature in the output file
\item {\ttfamily desc}\+: A description of how the feature is calculated
\end{DoxyItemize}

Note, if you know a priori the number of levels in your game, you may enter all your {\ttfamily perlevel} features as {\ttfamily per\+\_\+custom\+\_\+count} features. The only difference between the two is that {\ttfamily perlevel} features have a hard-\/coded prefix (\char`\"{}lvl\char`\"{}) and the number of levels is inferred from the max level in the database.
\end{DoxyItemize}
\end{DoxyItemize}

{\ttfamily aggregate}\+: This sub-\/element should be a dictionary mapping the names of features aggregated over a whole session to descriptions of how the features are calculated. {\ttfamily db\+\_\+columns} is used to ensure the raw csv file metadata contains descriptions of each database column. {\ttfamily events} are used to get names for the members of each kind of event so we can extract features (and create columns in the raw csv). {\ttfamily features} are used to ensure the processed csv file metadata contains descriptions of each feature, and to help document the features for whoever writes the actual feature extraction code. Below is a sample of J\+S\+ON schema formatting\+: 
\begin{DoxyCode}{0}
\DoxyCodeLine{\{}
\DoxyCodeLine{    "db\_columns": \{}
\DoxyCodeLine{        "id":"Unique identifier for a row",}
\DoxyCodeLine{    \},}
\DoxyCodeLine{}
\DoxyCodeLine{    "events": \{}
\DoxyCodeLine{        "ARROW\_MOVE\_RELEASE": \{}
\DoxyCodeLine{            "event\_custom":"string",}
\DoxyCodeLine{            "begin\_val":"float",}
\DoxyCodeLine{            "end\_val":"float"}
\DoxyCodeLine{        \}}
\DoxyCodeLine{    \},}
\DoxyCodeLine{}
\DoxyCodeLine{    "features": \{}
\DoxyCodeLine{        "perlevel": \{}
\DoxyCodeLine{            "totalSliderMoves":"slider moves across a given level",}
\DoxyCodeLine{        \},}
\DoxyCodeLine{        "per\_custom\_count": \{}
\DoxyCodeLine{            "questionAnswered" : \{"count" : 4, "prefix": "QA", "desc" : "The answer the user gave to a given question (or -1 if unanswered)"\},}
\DoxyCodeLine{        \},}
\DoxyCodeLine{        "aggregate": \{}
\DoxyCodeLine{            "avgSliderMoves" : "totalSliderMoves averaged over all levels",}
\DoxyCodeLine{        \}}
\DoxyCodeLine{    \}}
\DoxyCodeLine{\}}
\end{DoxyCode}



\begin{DoxyEnumerate}
\item Next, we need to create the feature extractor. This will be a Python class inheriting from the {\ttfamily Extractor} base class. By convention, the class should use the database app\+\_\+id as a prefix for the class name, but use Camel\+Case even if the app\+\_\+id is not formatted as such. e.\+g. For the \char`\"{}\+Wave\char`\"{} game, we would name the extractor {\ttfamily \mbox{\hyperlink{class_wave_extractor}{Wave\+Extractor}}} (as opposed to {\ttfamily app\+\_\+id + "Extractor} =$>$ {\ttfamily W\+A\+V\+E\+Extractor}). The {\ttfamily Extractor} subclass {\itshape must} implement the following functions\+:
\begin{DoxyItemize}
\item {\ttfamily \+\_\+\+\_\+init\+\_\+\+\_\+(self, session\+\_\+id, game\+\_\+table, game\+\_\+schema)}\+: At minimum, this function should call the super constructor. {\ttfamily session\+\_\+id} has the id of the session we are extracting data from, {\ttfamily game\+\_\+table} contains information about the database table, and {\ttfamily game\+\_\+schema} contains the data from the schema we defined in step 1. The super constructor initializes all features to have values of 0. If a different default value is preferred for any features, it would be a good idea to set those values here, after calling the super constructor.
\item {\ttfamily extract\+From\+Row(self, row\+\_\+with\+\_\+complex\+\_\+parsed, game\+\_\+table)}\+: This function is responsible for extracting feature data from a single database entry. The {\ttfamily row\+\_\+with\+\_\+complex\+\_\+parsed} should a row returned from the database, but with the item at {\ttfamily event\+\_\+data\+\_\+complex} already parsed into a Python dictionary from J\+S\+ON. If step 1 was completed correctly, the {\ttfamily event\+\_\+data\+\_\+complex} ought to match one of the {\ttfamily event}s from the J\+S\+ON schema. The {\ttfamily game\+\_\+table} holds information about the layout of the database table, as usual. It can be used to get items from the row at specific columns. This function should contain code to handle extraction from each {\ttfamily event} type. By convention, the actual extraction code should be split into separate private functions for each {\ttfamily event} type, so that we can get a cleanly-\/formatted {\ttfamily extract\+From\+Row} function, as below\+: \`{}\`{}\`{}python event\+\_\+type = event\+\_\+data\+\_\+complex\+\_\+parsed\mbox{[}\char`\"{}event\+\_\+custom\char`\"{}\mbox{]} if event\+\_\+type == \char`\"{}\+B\+E\+G\+I\+N\char`\"{}\+: self.\+\_\+extract\+From\+Begin(level, event\+\_\+client\+\_\+time) elif event\+\_\+type == \char`\"{}\+C\+O\+M\+P\+L\+E\+T\+E\char`\"{}\+: self.\+\_\+extract\+From\+Complete(level, event\+\_\+client\+\_\+time) elif event\+\_\+type == \char`\"{}\+S\+U\+C\+C\+E\+E\+D\char`\"{}\+: \section*{etc.}
\end{DoxyItemize}

\`{}\`{}\`{}

Each private function should then update feature values as needed\+: \`{}\`{}\`{}python def \+\_\+extract\+From\+Complete(self, level, event\+\_\+client\+\_\+time)\+: self.\+end\+\_\+times\mbox{[}level\mbox{]} = event\+\_\+client\+\_\+time self.\+features\mbox{[}\char`\"{}completed\char`\"{}\mbox{]}\mbox{[}level\mbox{]}\mbox{[}\char`\"{}val\char`\"{}\mbox{]} = 1 \`{}\`{}{\ttfamily  Also note that in general, we use}row\+\_\+with\+\_\+complex\+\_\+parsed\mbox{[}\char`\"{}event\+\_\+data\+\_\+complex\mbox{]}\mbox{[}\char`\"{}event\+\_\+custom\char`\"{}\mbox{]}\`{} to distinguish event types. Even if the original database entry did not have \`{}event\+\_\+custom\`{} as a part of the J\+S\+O\+N, the \`{}\+Proc\+Manager\`{} class will insert the value of the \`{}event\`{} database column as \`{}event\+\_\+custom\`{} in the \`{}row\+\_\+with\+\_\+complex\+\_\+parsed\mbox{[}\char`\"{}event\+\_\+data\+\_\+complex\mbox{]}{\ttfamily object, so there is at least \+\_\+some\+\_\+ way to tell what type of event is being processed. -\/}calculate\+Aggregate\+Features(self)\`{}\+: This function should use the values in its per-\/level and per-\/custom-\/count features to calculate the aggregate (across whole session) features. The code for calculating individual aggregate features may be broken into separate private functions if desired, although in practice most aggregate features can be calculated with just a couple lines of code apiece, so this is not usually necessary. Also, note that while the function is intended for aggregate feature calculation, in practice you may also need to calculate certain final values of per-\/level/count features here as well. For example, a feature which gives an average over a level can\textquotesingle{}t be calculated until we know all events for that level have been encountered. Hence, it would be recommended to accumulate a total in extract\+From\+Row, and calculate the average in calculate\+Aggregate\+Features. This function will generally be called just once during each extractor\textquotesingle{}s lifetime, after all rows corresponding to that session have been processed.
\item Next, we need to ensure Data\+To\+C\+SV knows what the possible games are. In the section of code (presently around line 70) dealing with loading of the schema file, we need to add a case to the if-\/elif-\/else block. It should check if the request object has a game\+\_\+id matching our new game, and if so, we must call the Schema constructor with the name of the schema file from step 1, and we must set {\ttfamily game\+\_\+extractor} to the Extractor class we created for the game. For example\+: 
\begin{DoxyCode}{0}
\DoxyCodeLine{if request.game\_id == "WAVES":}
\DoxyCodeLine{    game\_schema = Schema(schema\_name="WAVES.json")}
\DoxyCodeLine{    game\_extractor = WaveExtractor}
\end{DoxyCode}

\item Once those three steps are completed, the only thing left is to call Data\+To\+C\+SV with a request using the new game\textquotesingle{}s app\+\_\+id. This is usually done from main.\+py. 
\end{DoxyEnumerate}