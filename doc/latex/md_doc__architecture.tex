{\bfseries{This document gives a general overview of the various pieces of code in the opengamedata project.}}

\paragraph*{Basic Categories}

The project code has three primary parts\+:
\begin{DoxyEnumerate}
\item {\bfseries{Extraction}} code to extract features from raw game log data, retrieved from a database.
\item {\bfseries{Export}} code to perform feature extraction at scale, and export the data to files for distribution/archival.
\item {\bfseries{Evaluation}} code to run features of a game session through models (generated from past game data), and output model-\/based predictions.
\end{DoxyEnumerate}

\paragraph*{Location(s) of code by category}

The extraction code is primarily located in feature\+\_\+extractors folder, and makes use of data in the schemas folder. Export code mostly lives in the managers folder, and is invoked from main.\+py. Evaluation code is, at the time of writing, wrapped up in R\+T\+Server.\+py, which was part of a dashboard prototype. In future development, these should be separated somewhat. Evaluation and Export code make heavy use of the Extraction code.

\paragraph*{Code hierarchy and structure}

The general structure and data flow for the {\bfseries{exporter}} code is as follows\+:


\begin{DoxyEnumerate}
\item At the highest level, we have main.\+py. This is the file you run to set off the whole process. The code in main loads the program settings and constructs a \char`\"{}request\char`\"{} for the \mbox{\hyperlink{class_export_manager}{Export\+Manager}} module, which holds the core logic for the process. The details of the request are determined by the command-\/line args given to main.
\item The next level is Export\+Manager.\+py. This is where we do the high-\/level export processing. Given the export request, \mbox{\hyperlink{class_export_manager}{Export\+Manager}} will load additional data for the game specified by the request, retrieve data for that game from the database, and export both raw and processed csv files. The \char`\"{}additional data\char`\"{} includes\+:
\begin{DoxyItemize}
\item a \mbox{\hyperlink{class_game_table}{Game\+Table}}, which has information on the database table with game data, and \char`\"{}what\char`\"{} data is available.
\item a Schema, which has information on the structure of the stored game data itself. Specifically, this deals with data encoded within a string in a specific database column.
\end{DoxyItemize}

\mbox{\hyperlink{class_export_manager}{Export\+Manager}} makes use of two further \char`\"{}manager\char`\"{} classes (one for raw data, one for processed), which each provide functions for processing a single row, and writing out accumulated data to a file. This means \mbox{\hyperlink{class_export_manager}{Export\+Manager}} is only directly responsible for checking the number of sessions available for the game, and retrieving the rows for each session. It then passes each row to the \mbox{\hyperlink{class_raw_manager}{Raw\+Manager}} and the \mbox{\hyperlink{class_proc_manager}{Proc\+Manager}}, and periodically uses the managers to write the data to file. Finally, \mbox{\hyperlink{class_export_manager}{Export\+Manager}} maintains a J\+S\+ON file with data on all exported csv files.
\item a. Raw\+Manager.\+py is responsible for the details of processing each database row. This basically amounts to splitting any columns that contain J\+S\+ON data into multiple smaller columns, and ensuring any columns which may have commas in them (strings, sub-\/objects) are wrapped in quotes. Each time a row is processed, the corresponding csv line is added to a list, which may be written to file at any time. Note, \mbox{\hyperlink{class_raw_manager}{Raw\+Manager}} has a Clear\+Lines function, which should be (and presently is) called after using the Write\+Raw\+C\+S\+V\+Lines function to avoid writing duplicate lines.
\end{DoxyEnumerate}
\begin{DoxyEnumerate}
\item b. Proc\+Manager.\+py has similar responsibilities to \mbox{\hyperlink{class_raw_manager}{Raw\+Manager}}, however the details of processing a row are somewhat more complicated. Because a processed csv has features calculated from data, rather than simply being a lightly processed version of the raw data itself, there is a lot of variation between games. Thus, \mbox{\hyperlink{class_proc_manager}{Proc\+Manager}} further defers details to Extractor classes, each instance of which can maintain feature data for one session. \mbox{\hyperlink{class_proc_manager}{Proc\+Manager}} simply makes use of the {\ttfamily extract\+From\+Row} function provided by each Extractor, as well as an Extractor function for writing data to a given file. \mbox{\hyperlink{class_proc_manager}{Proc\+Manager}} also has a function for calculating aggregate features of all sessions, which should generally be used before writing data to csv.
\item The Extractor classes are based on Extractor.\+py, which is an abstract class defining a few functions that operate across all subclasses, and a few abstract functions which should be implemented by each subclass. The abstract functions are {\ttfamily extract\+From\+Row} and {\ttfamily calculate\+Aggregate\+Features}, which are the two points that vary from game to game. Each implementation of {\ttfamily extract\+From\+Row} should handle all possible event types (each row represents one event) for the given game. These events are documented in the game\textquotesingle{}s Schema file. An extractor stores feature data in a private class, which provides functions for accessing and modifying features by name (useful for aggregate features) or index (e.\+g. index may be level or question number).
\end{DoxyEnumerate}

The {\bfseries{evaluator}} code is somewhat wrapped up in the code for a real-\/time dashboard, at present. Future development should result in a cleaner separation of features. The list below is thus less of a hierarchy, more of a description of where each piece (and examples of use) can be found in the code.
\begin{DoxyEnumerate}
\item R\+T\+Server.\+py contains the function of interest for evaluating models, called Evaluate\+Log\+Reg\+Model (at present, we are only using logistic regression models; this should expand in the future). This function accepts a \char`\"{}model\char`\"{} and feature data for a given play session.
\item Models are stored as json files, which at the least must map the \char`\"{}code\char`\"{} name of a model to a corresponding dictionary. The model dictionary maps input parameter names to coefficient values, and at the least will map \char`\"{}display name\char`\"{} to a nicely formatted string used to display the name of the model in an end-\/user application. It is this dictionary that gets passed to Evaluate\+Log\+Reg\+Model.
\item The clearest example for using models is in get\+Predictions\+By\+Sess\+ID within R\+T\+Server.\+py. This uses a model file that adds an extra level to the J\+S\+ON, separating models by the game level for which they are intended.
\end{DoxyEnumerate}

In the future, the intent is to have a separate module for model evaluation, which would allow us to do things like run models on test data, and export results for evaluation of model effectiveness, or creation of model benchmarks. 